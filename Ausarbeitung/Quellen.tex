\newpage 


\thispagestyle{empty}


 \section*{Quellen}



\begin{itemize}

  \item \makeSource {Indrani Medhi, Aman Sagar, and Kentaro Toyama}
					{Text-Free User Interfaces for Illiterate and Semi Literate Users}
					{2006}
					{eine-pdf-(Ursprung-unbekannt)}
	
	\item \makeSource {Judith Ramey - University of Washington}
					{Use of Mobile Phones by Non-literate and Semiliterate People: A Systematic Literature Review}
					{2011}
					{eine-pdf-(Ursprung-unbekannt)}
					
	\item \makeSource {INDRANI MEDHI, SOMANI PATNAIK, EMMA BRUNSKILL, S. N. NAGASENA GAUTAMA and WILLIAM THIES, KENTARO TOYAMA}
					{Designing Mobile Interfaces for Novice and Low-Literacy Users}
					{2011}
					{eine-pdf-(Ursprung-unbekannt)}

	\item \makeSource {William M. Gribbons}
					{Universal Accesibility and Low-Literacy Populations}
					{Implications for Human-Computer, Interactions Desing and Research Methods}
					{Auflage unbekannt}
					{Auszug: The Human-Computer Interaction Handbook: Fundamentals, Evolving Technologies and Emerging Applications}

	\item \makeSource {Andrea Schultens}
						{Analphabeten}
						{Stand vom 01.06.2009}
						{http://www.planet-wissen.de/alltag_gesundheit/lernen/analphabeten/index.jsp}
										

	\item \makeSource {Max Wüstehube, Internatsschule Schloss Hansenberg, Geisenheim }
						{Abgemalte Aufgaben}
						{Stand vom 18.07.2008}
						{http://www.faz.net/aktuell/gesellschaft/jung/jugend-schreibt/probleme-von-analphabeten-abgemalte-aufgaben-1665659.html}									

	\item \makeSource {Maria Gerber}
						{Wie Analphabeten in Deutschland unerkannt bleiben}
						{Stand vom 20.09.2010}
						{http://www.welt.de/wissenschaft/article9750415/Wie-Analphabeten-in-Deutschland-unerkannt-bleiben.html}	

	\item \makeSource {Unbekannt }
						{759 Millionen Menschen sind Analphabeten}
						{Stand vom 08.9.2010}
						{http://www.epo.de/index.php?option=com_content&view=article&id=6455}										

	\item \makeSource {Karin Jäger}
						{Das neue Leben der Analphabeten }
						{Stand vom 12.10.2012}
						{http://www.dw.de/das-neue-leben-der-analphabeten/a-16197988}										

	\item \makeSource {Myriam Schäfer}
						{Mantel des Schweigens}
						{Stand vom 08.09.2012}
						{http://www.freitag.de/autoren/myriam/bildung-fuer-alle}										

	\item \makeSource {Sabine Kuhn-Behrenbeck}
						{Dumm bin ich weiß Gott nicht}
						{Stand unbekannt}
						{http://www.mobile-elternmagazin.de/wireltern/partnerschaft_familie/details?k_onl_struktur=385577&k_beitrag=523759}			

										

	\item \makeSource {Jörg Bock}
						{Was passiert im Gehirn? - AD(H)S, Legasthenie, Dyskalkulie und Hochbegabung}
						{Stand 19.4.2008}
						{http://www.google.de/url?sa=t&rct=j&q=&esrc=s&source=web&cd=5&ved=0CE4QFjAE&url=http\%3A\%2F\%2Fwww.landeselternrat-sachsen.de\%2Ffileadmin\%2Fler\%2Fdaten\%2F01ler\%2F04sitzungen\%2F92--06-07_07-08\%2F080419_Vortrag-Bock.pdf&ei=vKn9UPsNzcmzBuTXgbAO&usg=AFQjCNE37lYgvH9tU_zk5QKyV_bgx-wX-Q&bvm=bv.41248874,d.Yms}
	
	\item \makeSource {Bundesverband Alphabetisierung und Grundbildung e.V.}
						{-Kein direkter Titel, die ganze Internetseite-}
						{Stand -}
						{http://www.alphabetisierung.de/}
						
	\item \makeSource {-}
						{Invisque - INteractive VIsual Search and QUery Environment}
						{Stand -}
						{http://www.invisque.com/index.html}
	\item \makeSource {Microsoft-Server 2007 Installation}
						{Microsoft-Server 2007 Installation}
						{Stand 14.2.2013}
						{http://i.technet.microsoft.com/dynimg/IC22550.gif}
	\item \makeSource {Google-Suche  \glqq alfabeht\grqq}
						{Google-Suche \glqq alfabeht\grqq}
						{Stand 14.2.2013}
						{www.google.de}
	
\end{itemize}

