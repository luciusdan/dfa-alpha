\section{Design}

Beim Design für Analphabeten gibt es einige Ansätze, wie man Dinge wie bspw. Software gestalten kann, damit auch Menschen, die nicht lesen und schreiben können, diese Dinge verwenden können.\\
Hierbei ist es natürlich wichtig herauszufinden, wie diese Menschen Dinge wahrnehmen auffassen und wie ihre Herangehensweise beim erlernen neuer Dinge ist. Um dies herauszufinden ist es unter anderem in erster Linie wichtig Studien abrufen zu können, in denen soetwas bereits gemacht wurde. Dabei muss man in den meisten Fällen jedoch auf Studien zurückgreifen, die sich mit primärem Analphabetismus auseinandersetzen, da diese Studien in den meisten Fällen in Indien und den Phillipinen gemacht wurden, bei Menschen, die das Lesen und Schreiben niemals erlernt haben. Aber auch diese werden vorraussichtlich ähnliche Methoden wie funktionale Analphabeten verwenden um zu lernen.\\
