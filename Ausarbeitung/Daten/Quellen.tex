\newpage 



\thispagestyle{empty}



 \section*{Quellen}




\begin{itemize}
% -- Quellenergaenzung aus Designkapitel ----------------------------
 	 \item \makeSource {\label{W.M.Gribbons01}William M. Gribbons}
					{Universal Accesibility and Low-Literacy Populations, Implications for Human-Computer, Interactions Desing and Research Methods}
					{2012}
					{The Human-Computer Interaction Handbook: Fundamentals, Evolving Technologies and Emerging Applications}

	 \item \makeSource {\label{up_goer_five_01}Jason Major}
				{This is Awesome: U.S. Space Team’s “Up Goer Five”}
				{5.6.2013}
				{http://www.universetoday.com/98411/this-is-awesome-u-s-space-teams-up-goer-five/}


	 \item \makeSource {\label{ mc_serv_01}Microsoft}
						{Microsoft-Server 2007 Installation}
						{Stand : 14.2.2013}
						{http://i.technet.microsoft.com/dynimg/IC22550.gif} 

	 \item \makeSource {\label{google01}Google Inc.}
						{Google Suchmaschiene}
						{Stand : 5.6.2013}
						{www.google.de}

	 \item \makeSource {\label{alphaBund01}Bundesverband Alphabetisierung}
						{Bundesverband Alphabetisierung Website}
						{Stand : 14.2.2013}
						{http://www.alphabetisierung.de/}

	 \item \makeSource {\label{Invisque01}Invisque - INteractive VIsual Search and QUery Environment}
						{Stand: 5.6.2013}
						{http://www.invisque.com/index.html}{}

	 \item \makeSource {\label{MedhiSagarKentaro01}Indrani Medhi, Aman Sagar, and Kentaro Toyama}
					{Text-Free User Interfaces for Illiterate and Semi Literate Users}
					{2006}
					{}
% ---------------------------------



 	 \item \makeSource {Indrani Medhi, Aman Sagar, and Kentaro Toyama}

					{Text-Free User Interfaces for Illiterate and Semi Literate Users}

										{2006}

										{}



	\item \makeSource {Andrea Schultens}

										{Analphabeten}

										{Stand vom 01.06.2009}

										{http://www.planet-wissen.de/alltag_gesundheit/lernen/analphabeten/index.jsp}

										


	\item \makeSource {Max Wüstehube, Internatsschule Schloss Hansenberg, Geisenheim }

										{Abgemalte Aufgaben}

										{Stand vom 18.07.2008}

										{http://www.faz.net/aktuell/gesellschaft/jung/jugend-schreibt/probleme-von-analphabeten-abgemalte-aufgaben-1665659.html}									


	\item \makeSource {Maria Gerber}

										{Wie Analphabeten in Deutschland unerkannt bleiben}

										{Stand vom 20.09.2010}

										{http://www.welt.de/wissenschaft/article9750415/Wie-Analphabeten-in-Deutschland-unerkannt-bleiben.html}	


	\item \makeSource {Unbekannt }

										{759 Millionen Menschen sind Analphabeten}

										{Stand vom 08.9.2010}

										{http://www.epo.de/index.php?option=com_content&view=article&id=6455}										


	\item \makeSource {Karin Jäger}

										{Das neue Leben der Analphabeten }

										{Stand vom 12.10.2012}

										{http://www.dw.de/das-neue-leben-der-analphabeten/a-16197988}										


	\item \makeSource {Myriam Schäfer}

										{Mantel des Schweigens}

										{Stand vom 08.09.2012}

										{http://www.freitag.de/autoren/myriam/bildung-fuer-alle}										


	\item \makeSource {Sabine Kuhn-Behrenbeck}

										{Dumm bin ich weiß Gott nicht}

										{Stand unbekannt}

										{http://www.mobile-elternmagazin.de/wireltern/partnerschaft_familie/details?k_onl_struktur=385577&k_beitrag=523759}			


										


	\item \makeSource {Jörg Bock}

										{Was passiert im Gehirn? - AD(H)S, Legasthenie, Dyskalkulie und Hochbegabung}

										{Stand 19.4.2008}

										{http://www.google.de/url?sa=t&rct=j&q=&esrc=s&source=web&cd=5&ved=0CE4QFjAE&url=http\%3A\%2F\%2Fwww.landeselternrat-sachsen.de\%2Ffileadmin\%2Fler\%2Fdaten\%2F01ler\%2F04sitzungen\%2F92--06-07_07-08\%2F080419_Vortrag-Bock.pdf&ei=vKn9UPsNzcmzBuTXgbAO&usg=AFQjCNE37lYgvH9tU_zk5QKyV_bgx-wX-Q&bvm=bv.41248874,d.Yms}


\end{itemize}


