\section{Gr"unde f"ur Analphabetismus}


Hier an dieser Stelle fragt man sich nat"urlich "'\textit{Wie kann es sein, dass so viele Menschen niemals Lesen und Schreiben gelernt haben?}"` Immerhin gibt es besonders in Deutschland die Schulpflicht, wo einem Lesen und Schreiben beigebracht wird. Wie kann es da also sein, dass ganze 6\% der deutschen Bev"olerung allein Analphabeten sind? Hat das Bildungssystem versagt? \\

\subsection{Bildung}
Die Gr"unde hierf"ur sind zum gro"sen Teil subtiler als man denken w"urde. Hier spielen zudem viele Faktoren eine Rolle wie Familie, Freunde und die einfache Mentalit"at eines Menschen. Um direkt ein kleines Beispiel f"ur den Bereich der Mentalit"at zu nennen, gehen wir zur"uck in die Anf"ange unserer Schulzeit. Wir alle kennen das Verhalten unter Kindern, die sich gegenseitig h"anseln, aufziehen und h"aufig andere unterdr"ucken, nur um dazu zugeh"oren. Wer wurde bspw. nicht ausgelacht und sch"amte sich, wenn er falsche Antworten gab? Nun stelle man sich einen Sch"uler vor, der ein wenig langsamer begreift als seine Mitsch"uler. Wahrlich keine Seltenheit. Jeder Mensch lernt in seinem eigenen Tempo und jeder hat seine eigenen Techniken um sich Dinge zu merken. Viele lernen sehr schnell, andere brauchen doppelt so lange wie jene, oder manchmal sogar noch l"anger. Das bedeutet bei weitem nicht, dass diese Menschen lernbehindert oder dumm sind. Es bedeutet nur, dass sie eine andere Art und Weise haben Dinge aufzunehmen, abzuspeichern und Bez"uge herzustellen. Wenn nun also ein Kind in der Schule laut vorlesen muss, um zu zeigen, was es gelernt hat, es ihm aber schwer f"allt, da dieses Kind bisher die Grundz"uge noch nicht ganz erfassen konnte und deshalb bei diesem Versuch laut stottert, wird es mit Sicherheit sehr laut von den anderen Kindern ausgelacht und geh"anselt. dies nagt an der Mentalit"at der Kinder. Durch solche Situationen wird das Selbstvertrauen geschmählert und die Kinder fallen im Stoff weiter zurück. Da Lehrer in der Regel auch über solche Zustände nicht aufgeklärt werden, wissen sie mi solchen Situationen wenig anzufangen und fördern diese Kinder auch nicht entsprechend. Die Folge daraus ist, dass die Kinder mit dem Stoff auf der Strecke bleiben und aus genau jenem Grund, dass sie nicht lesen können keine Chance mehr haben richtig hinterherzukommen.\\
Was bleibt diesen Kindern nun also noch übrig? Die richtige Entscheidung wäre die Eltern oder Freunde um Hilfe zu bitten. Jedoch gehen nicht alle diesen Weg. Einige Kinder schämen sich so sehr, dass sie alles versuchen um die Tatsache, dass sie nicht lesen und schreiben können, zu vertuschen, anstatt sich mit diesem Problem richtig auseinanderzusetzen. Und somit geht es los, dass diese Kinder sich durch die Schule hindurchmogeln, wovon viele es auch tatsächlich schaffen einen Abschluss zu erhalten. Und somit wäre er geboren der erwachsene funktionale Analphabet. \\

Dieses Beispiel soll nur eine von vielen Möglichkeiten illustrieren, wieso so viele erwachsene Menschen in Ländern, in denen Schulpflicht herrscht Analphabeten sind. Das eben beschriebene Beispiel gleicht dem Erfahrungsbericht eines Mannes, der über sein Problem gesprochen hat und es so erlebte. Auch wenn noch weitere Faktoren zu dieser Geschichte hinzugefügt werden müssten, kann man hier einen groben Einblick erhalten, wie es zu solch einem unglaublichen Szenario kommen kann.\\

Weitere wichtig ernstzunehmende Probleme lassen sich bspw. mit Küstermanns Zitat vom Neuköllner Verein "'Lesen \& Schreiben e.V."' erläutern.

\begin{quote}
	Ein Hauptproblem ist, dass Muttersprachler gesetzlich
	kein Anrecht auf nachträglichen Schriftspracherwerb
	haben, anders als Bürger mit Migrationshintergrund. Hier
	gibt es bisher für Deutsche, die an den Rand der
	Gesellschaft gedrängt werden, kaum Angebote, die
	finanziert werden.
\end{quote}


\subsection{Legasthenie}

Legasthenie ist eine Erbkrankheit, bei der eine Störung der visuellen und auditiven Wahrnehmungsverarbeitung auftritt. Menschen mit dieser Störung leiden nicht an mangelndem Intellekt oder an motorischen Einschränkungen, stattdessen ist einfach ihr Sehen und/oder ihr Hören beeinträchtigt. Dadurch können diese Menschen Text nciht richtig entziffern und können auch gesprochene Worte einfach akkustisch nicht richtig verstehen. Experten sind sich uneinig, ob Legasthenie heilbar ist, oder nicht. In jedem Falle wird bei einer frühzeitigen Diagnose dazu geraten, das Kind einem speziellen Training zu unterziehen, damit es die sehenden und akkustischen Eigenschaften, die nicht richtig ausgebildet sind, nach entwickelt. Um ein näheres Gefühl und Verständnis dafür zu bekommen, wieso ein Legastheniker solch enorme Schwierigkeiten hat Text zu erkennen und zu entziffern, soll folgende Grafk dienen.\\

\includegraphics[width=1.00\textwidth]{Daten/legastheneWahrnehmung.jpg}




