\section{Gr"unde f"ur Analphabetismus} \label{sec:reasons}

Hier stellt sich die Frage "'\textit{Wie kann es sein, dass so viele Menschen niemals Lesen und Schreiben gelernt haben?}"` Immerhin gibt es in Deutschland die Schulpflicht, wo dem Menschen Lesen und Schreiben beigebracht wird. Wie kann es da also sein, dass ganze 6\% der deutschen Bev"olkerung Analphabeten sind? Hat das Bildungssystem versagt? \\


\subsection{Bildung}

Die Antworten auf die Frage, wieso so viele Menschen in Deutschland Analphabeten sind, sind zum gro"sen Teil einfacher als gedacht. Hier spielen viele Faktoren eine Rolle wie etwa Familie, Freunde und die Mentalit"at eines Menschen. Um direkt ein kleines Beispiel f"ur den Bereich der Mentalit"at zu nennen, gehen wir zur"uck in die Anf"ange der Schulzeit. Dabei wollen wir den folgenden Teil auf den Erfahrungsbericht eines Analphabeten stützen, welchen wir später noch genauer behandeln werden.
\footnoteSource {Sabine Kuhn-Behrenbeck}
				{Dumm bin ich weiß Gott nicht}
				{18.01.2013 }
				{http://www.mobile-elternmagazin.de/wireltern/partnerschaft_familie/details?k_onl_struktur=385577&k_beitrag=523759}
In diesem Erfahrungsbericht geht es um Bernd Dahler, der zu seiner Zeit eine schwierige Phase in der Schule hatte. Er war starker Stotterer und wenn er etwas vorlesen sollte in der Schule, wurde er von den anderen Kindern ausgelacht, weshalb er versuchte, solche Situationen zu vermeiden. Außerdem ist Bernd Linkshänder und wurde von seinem Lehrer gezwungen, mit der rechten Hand zu schreiben, weshalb es ihm zusätzlich noch schwerer gefallen ist. Als er im Anschluss die Grundschule verließ und auf die Hauptschule kam, mogelte Bernd sich nur noch durch. Wurden Klassenarbeiten geschrieben, blieb er zuhause und spielte krank. Viel vom Unterrichtsstoff bekam er mit und merkte sich alles auf seine eigene Weise. Er bestand die Hauptschule schließlich, erhielt allerdings von den Lehrern in seinem Zeugnis auch den Vermerk, dass er nicht lesen und schreiben kann.\\
All dies muss aber auch noch unter anderen Aspekten betrachtet werden.
Nicht alle Menschen lernen im gleichen Tempo, weshalb einige mehr Zeit und mehr Aufmerksamkeit benötigen, um ihre Aufgaben zu meistern. Wird denen, die langsamer lernen, die entsprechende Aufmerksamkeit in Bezug auf die Betreuung nicht gewährt, verlieren sie den Anschluss an den Lehrplan. Sie schaffen es teils nicht, den "`verpassten"' Lehrstoff nachzuholen, da sie versuchen, den neuen Themen zu folgen und dabei schließlich gänzlich den Faden verlieren. Dies deutet allerdings keineswegs darauf hin, dass die entsprechenden Personen lernbehindert sind. Es zeigt sich eher, dass nicht alle Menschen auf die gleiche Art und Weise Dinge aufnehmen und Bezüge herstellen. Aus diesem Grund ist es notwendig, dass verschiedene Lehrmethoden auf größere Gruppen angewendet werden.\\

Dieses Beispiel soll nur einen von vielen Gründen zeigen, warum so viele erwachsene Menschen in Ländern, in denen Schulpflicht herrscht, Analphabeten sind. \\

Weitere sehr ernstzunehmende Probleme lassen sich bspw. mit Küstermanns Zitat vom Neuköllner Verein "'Lesen \& Schreiben e.V."' erläutern.


\begin{quote}
	"`Ein Hauptproblem ist, dass Muttersprachler gesetzlich
	kein Anrecht auf nachträglichen Schriftspracherwerb
	haben, anders als Bürger mit Migrationshintergrund. Hier
	gibt es bisher für Deutsche, die an den Rand der
	Gesellschaft gedrängt werden, kaum Angebote, die
	finanziert werden."'
\footnoteSource {Oliver Ohmann}
				{Zwei Berliner über ihr Leben ohne Worte}
				{02.02.2013}
				{http://www.bz-berlin.de/aktuell/berlin/zwei-berliner-ueber-ihr-leben-ohne-worte-article1133404.html}
\end{quote}


Zudem sollte noch erwähnt werden, wie Analphabeten in etwa auf das Bildungssystem abgebildet werden können.\\
Studien zufolge haben etwa 

\begin{itemize}
	\item 19\% der Analphabeten keinen Schulabschluss
	\item 48\% erreichten einen Hauptschulabschluss
	\item 19\% haben einen Realschulabschluss
	\item und 12\% erreichten sogar einen höheren Bildungsabschluss als das Abitur
\end{itemize}

Leider wurden die hier fehlenden 2\% nicht weiter erwähnt.
\footnoteSource {Prof. Dr. Anke Grotlüschen, Dr. Wibke Riekmann}
				{Ein Mensch wie Du und ich}
				{02.06.2013}
				{http://blogs.epb.uni-hamburg.de/leo/files/2012/09/2012-19-10-leo-f\%C3\%BCr-Fachtagung-Bad-Wildungen.pdf.}



\subsection{Legasthenie}

Legasthenie ist eine Erbkrankheit, bei der eine Störung der visuellen und auditiven Wahrnehmungsverarbeitung auftritt. Menschen mit dieser Störung leiden nicht an mangelndem Intellekt oder an motorischen Einschränkungen, stattdessen ist einfach ihr Sehen und/oder ihr Hören beeinträchtigt. Dadurch können diese Menschen Texte nicht richtig entziffern und können auch gesprochene Worte akustisch nicht richtig verstehen. Viele Quellen sind sich einig, dass Legasthenie nicht heilbar ist, dass dem Ganzen jedoch mit frühem Training entgegen gewirkt werden kann. 
\footnoteSource{Andrea Schultens}
				{Wie unterscheiden sich Analphabetismus und Legasthenie?}
				{02.06.2013}
				{http://www.planet-wissen.de/alltag_gesundheit/lernen/analphabeten/wissensfrage.jsp}
Auf der anderen Seite gibt es aber auch wieder andere Quellen, die sich sicher sind, dass Legasthenie geheilt werden kann.
\footnoteSource{Prof. Dr. Renate Valtin, Dr. Ilona Löffler}
				{Legasthenie ist heilbar}
				{02.06.2013}
				{http://www.dgls.de/download/category/4-tagungen.html?download=44:r-valtin-und-i-loeffler-legasthenie-ist-heilbar}

 In jedem Falle wird bei einer frühzeitigen Diagnose dazu geraten, das Kind einem speziellen Training zu unterziehen, damit es die visuellen und auditiven Fähigkeiten, die nicht richtig ausgebildet sind, weiterentwickelt. Um ein größeres Verständnis dafür zu bekommen, wieso ein Legastheniker solch enorme Schwierigkeiten hat, Texte zu erkennen und zu entziffern, soll folgende Grafik dienen.\\

\begin{figure}[h]
	\centering
		\includegraphics[width=1.00\textwidth]{Daten/legastheneWahrnehmung.jpg}
	\caption{Legasthene Wahrnehmung}
	\url{http://www.landeselternrat-sachsen.de/fileadmin/ler/daten/01ler/04sitzungen/92--06-07_07-08/080419_Vortrag-Bock.pdf}
	\label{fig:legastheneWahrnehmung}
\end{figure}