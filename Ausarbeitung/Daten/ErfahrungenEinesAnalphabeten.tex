\section{Erfahrungen eines Analphabeten}


Nach dem Erfahrungsbericht eines Analphabeten, belegen sich viele der Tatsachen die im vorherigen Abschnitt \ref{sec:reasons} unter dem Punkt Bildung genannt wurden. Dabei geht es um Bernd Dahler, der zu der Zeit, als er dieses Interview gab, 36 war.
\footnoteSource{Sabine Kuhn-Behrenbeck}
				{Dumm bin ich weiß Gott nicht}
				{18.01.2013}
				{http://www.mobile-elternmagazin.de/wireltern/partnerschaft_familie/details?k_onl_struktur=385577&k_beitrag=523759}\\


\begin{figure}[h]
	\centering
		\includegraphics[scale=0.80]{Daten/BerndDahler.jpg}
	\caption{Bernd Dahler:} 
	\url{http://www.mobile-elternmagazin.de/wireltern/partnerschaft_familie/details?k_onl_struktur=385577&k_beitrag=523759}
	\label{fig:BerndDahler}
\end{figure}

Hier hatte Bernd unter anderem geschildert, dass es weder f"ur ihn, noch f"ur seine Eltern leicht war. Er war das 2. j"ungste Kind in einer Familie mit 11 Kindern. In einer derart gro"sen Familie l"asst sich nachvollziehen, dass die Kinder gr"o"stenteils sich selbst "uberlassen waren und auf die Bed"urfnisse der einzelnen Individuen nicht gro"s eingegangen werden konnte. Dies ist einer der Gr"und daf"ur, dass Bernd in seiner Kindheit keine zus"atzliche Unterst"utzung aus der Familie bekam.\\

In der Schule hatte er bereits am Anfang Probleme beim Lesen, da er stark stotterte, wofür er ausgelacht wurde, wenn er laut vorlesen musste. Au"serdem wurde er von seinen Lehrern immerzu gen"otigt, mit der rechten Hand zu schreiben, obwohl er Linksh"ander ist. Und wie einige Studien belegen, soll dies der Entwicklung der Kinder Schaden zufügen können.
\footnoteSource{Johanna Barbara Sattler }
				{Linkshänder und umgeschulte Linkshänder in der Ergotherapie }
				{02.06.2013}
				{http://www.lefthander-consulting.org/deutsch/Praxisergo.htm}\

Da Bernd den Anschluss nicht behielt und es nicht fertig brachte das Lesen im gleichen Tempo wie seine Mitsch"uler zu meistern, blieb er auf der Strecke. Daher spielte er meistens krank, wenn Klassenarbeiten geschrieben werden sollten. Den Lehrstoff an sich versuchte er mit schierer Aufmerksamkeit im Untericht aufzunehmen, was für ihn reichte, um ihm einen Schulabschluss zu erhalten. Selbstverst"andlich waren die Lehrer "uber seinen Zustand informiert und vermerkten seine Lese-/Schreibschw"ache auch in seinem Zeugnis.\\

 Als er schlie"slich mit der Schule fertig war, versuchte er einen Job zu finden, in dem es keine starke Anforderungen an die Lese- bzw. Schreibf"ahigkeit gab. Somit kam er zu einer Ausbildung als Galvaniseur, welcher ein handwerklicher Beruf ist, in dem zumeist mit Metallen gearbeitet wird. Diese Ausbildung dauerte f"ur ihn 3 Jahre und er durfte am Ende seine Abschlusspr"ufung als reine m"undliche Pr"ufung absolvieren.\\

In seinem Privatleben gibt es niemanden unter seinen Freunden, der wei"s, dass er nicht lesen kann. Um dies unter anderem vor ihnen zu verbergen, bestellt er bspw. in Restaurants jedes mal dasselbe "`Wienerschnitzel mit Pommes"'. Wenn er den Stadtplan lesen musste, hatte er einfach seine Brille vergessen und bat andere darum, ihm den Plan vorzulesen.\\

Sein Allgemeinwissen hat er auf einen guten Stand gebracht, indem er sich vieles was er im Fernsehen sieht einfach merkt, indem er Dokumentationen und "ahnliches schaut. Dumm ist Bernd nicht, nur hat er es leider in seiner Ausblidung aufgrund mangelnder Unterst"utzung nicht geschafft, das Lesen zu erlernen.

