\section{Zusammenfassung}
Bei dem Thema Analohabetismus stellte die Ähnlichkeit zum normalen Benutzer ein Hinderniss dar.
Darum haben wir auch zuerst gezeigt wie häufig die Lese und Rechtschreibschwäche vorkommt und was es für die Betroffenen im Altäglichen leben darstellt. Dadurch sollte es deutlich geworden sein, wie wichtig es ist sich mit der Zielgruppe zu beschäftigen, besonders für weniger soziale Länder. Wir haben verschiedene Möglichkeiten vorgestellt, um den Analphabeten den Zugang zu erleichtern, welche auch über das vermeiden vom ganzen Text hinausgingen und haben Verbindungen zu anderen Einschränkungen gezogen. Um diese Designs zu veranschaulichen wurden verschiedene, mehr oder weniger effektive, Beispiele und wie sie mit dem Thema Analphabetismus arbeiten gezeigt. Leider führen die meisten Anwendungen erst über das Betriebssystem, welches bereits eine Hürde darstellt.\\\\
Wir hoffen, dass nun klar ist, was es bedeutet für Analphabeten eine Anwendung zu schreiben.