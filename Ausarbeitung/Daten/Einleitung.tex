\section{Einleitung}



In unserem derzeitigen Zeitalter der Kommunikation über technische Medien kommen wir zu keinem Tag drum herum einen Text vor uns zu sehen. Der Text an sich ist zu einer allgemeinen abstrakten Kommunikationsschnittstelle geworden, der wir Menschen uns bedienen und mit der wir uns exakt untereinander und sogar mit Maschinen austauschen können. Text dient dazu um Informationen zu speichern/archivieren, sich selbst nach außen hin zu präsentieren, Geschichten zu erzählen und vieles mehr.\\


Und doch gibt es viele Menschen, die in Deutschland und auf dem Rest der Welt, weder lesen noch schreiben können. In Deutschland alleine gibt es ca. 7,5 Millionen Menschen, das sind etwa 6\% der Bevölkerung, denen diese wichtige Fähigkeit versagt ist. Auf der ganzen Welt hingegen gibt es ungefähr 775 Millionen Menschen die das Lesen und Schreiben nicht beherrschen. \\


 Wie also sollen diese Menschen in der heutigen Welt zurecht kommen? Befindet sich einer aus dieser Gruppe bspw. in einem Restaurant, wie soll es ihm möglich sein, etwas von der Karte zu bestellen? Wie soll so eine Person herausfinden, welchen Bus sie nehmen muss, um wieder nachhause zu kommen? Wie soll sie sich bewerben um einen Job zu kriegen, wenn sie ihren eigenen Lebenslauf nicht schreiben kann? Diese und vielen weiteren Problemen sind diese Menschen Tag für Tag ausgesetzt. Es ist ein ständiger Kampf für sie in der Welt da draußen, die von Texten berherrscht wird. Deshalb wollen wir in dieser Ausarbeitung darauf eingehen, wie man Designs so auslegen kann, dass selbst Leute, die des Lesens nicht mächtig sind, mit diesen Objekten zurecht kommen.\\
 Wir werden uns dem ganzen in langsamer Form nähern, indem wir einige Wissenswerte Themen behandeln werden, die es über diese Menschen zu erwähnen gibt und arbeiten uns anschließend, wenn wir soweit sind, dass wir uns ein wenig in diese Menschen einfühlen können, zu der Herangehensweise hervor, wie Designs entwickelt werden sollten und geben auch einige Beispiele, wie diese umgesetzt wurden.


\subsection{Definition}



Die eben erwähnten Menschen, die nicht lesen und nicht schreiben können werden im Volksmund "`Analphabeten"' genannt. Es stimmt jedoch nicht, dass diese Leute überhaupt nicht lesen und schreiben können. Der Großteil der Analphabeten kann in einem begrenzten Maße lesen und ist dazu in der Lage einige Wörter zu schreiben.\\


Somit wäre eine gängige Definition für Analphabeten:\\


Menschen die das Lesen und Schreiben nicht, bis nur teilweise beherrschen.


\subsection{Arten des Analphabetismus}

Der Analphabetismus wird von Fachleuten in verschiedene Kategorien eingeteilt. Da die Menschen in der Regel aus verschiedenen Gründen nicht richtig lesen und schreiben können scheint eine solche Unterteilung auch durchaus sinnvoll:

\begin{itemize}
	\item primärer Analphabetismus


				\begin{itemize}


					  \item Ein primärer Analphabet ist einer, der das Lesen und Schreiben niemals gelernt hat.


				\end{itemize}


	\item sekundärer Analphabetismus


				\begin{itemize}


					  \item Ein sekundärer Analphabet ist einer, der das Lesen und Schreiben ursprünglich mal erlernt hat, es aber wieder verlernte, da er diese Fähigkeit nie einzusetzen brauchte.


				\end{itemize}


	\item Semianalphabetismus


				\begin{itemize}


					  \item Ein Semianalphabet ist man dann, wenn man des Lesens, aber nicht des Schreibens mächtig ist.


				\end{itemize}


	\item funktionaler Analphabetismus


				\begin{itemize}


					  \item Ein funktionaler Analphabet ist man dann, wenn man einzelne Worte lesen und schreiben kann, es einem aber nicht möglich ist längere vollständige Texte zu erfassen und zu verstehen.\\
						(Dieser Typ des Analphabetismus macht den größten Teil der Analphabeten aus und ist auch in Deutschland besonders stark vertreten.)


				\end{itemize}


\end{itemize}


