\documentclass{handout}

\usepackage[utf8x]{inputenc}
\usepackage[T1]{fontenc}
\usepackage[ngerman]{babel}
\usepackage{amsmath,amsfonts,amssymb}
\usepackage{lmodern}
\usepackage{marvosym}
\usepackage{graphicx}

\SetInstructor{Pascal Kn"uppel, Jan-Bernd Vosteen, Dirk Evers}
\SetCourseTitle{Design for All}
\SetHandoutTitle{Design für Analphabeten}
\SetDueDate{WISE 2012/13}

\begin{document}
\section{Analphabetismus}
	\textbf{Was ist Analphabetismus?:}
	\begin{description}
		\item [-] Das nicht- bis nur teilweise Beherrschen vom Lesen und Schreiben
		\item [-] Weltweit ca. 775 Millionen Analphabeten (Stand 2012)
		\item [-] Deutschland ca. 7,5 Millionen Analphabeten (Stand 2012 ca. 6\%)
	\end{description}

	\textbf{Arten des Analphabetismus:}
		\begin{itemize}
    			\item primärer Analphabetismus
    				\begin{itemize}
    					\item Wenn man das Lesen und Schreiben nie gelernt hat
    				\end{itemize}
    				
			\item sekundärer Analphabetismus				
    				\begin{itemize}
    					\item Wenn das Lesen und Schreiben wieder verlernt wurde
    				\end{itemize}
				
			\item Semianalphabetismus
				\begin{itemize}
    					\item Wenn man lesen, aber nicht schreiben kann
    				\end{itemize}
    				
			\item funktionaler Analphabetismus
				\begin{itemize}
    					\item Wenn man einzelne Worte versteht, aber mit langen Texten und deren Zusammenhängen massive Schwierigkeiten hat
    				\end{itemize}
		\end{itemize}

\begin{tabular}{ll}
 \parbox{7cm}{
		\textbf{Ängste vieler Analphabeten:}\\   
 	\begin{itemize}
		\item Ablehnung
		\item Als dumm bezeichnet zu werden
		\item Verspottung
		\item vor Bestrafung
			\begin{itemize}
				\item als Kind bspw. in der Schule
				\item als Erwachsener bspw. durch Jobverlust
			\end{itemize}
			
	\end{itemize}} &
 \parbox{7cm}{
	\textbf{Studie zur Ausbildung von Analphabeten:}\\
	\begin{itemize}
		\item 19\% keinen Schulabschluss. 
		\item 48\% haben einen niedrigen Schulabschluss. 
		\item 12\% verfügen über einen hohen Schulabaschluss. 
		\item 21\% keine Angabe.
	\end{itemize}
}
\end{tabular}
\section{Design}
\begin{tabular}{| l | l | }
\hline
Lesen & Merken \\
\hline
\hline
- Sprachausgabe ermöglichen     &  - keine Ablenkung\\
- einheitliche Schrift                     & - eine Aufgabe zu gleich\\
- deutliche Schriftart                   & - vermeiden von Wiedersprüchen\\
- Illusstrationen                           & - Informationen reduzieren\\
- Kontext wiedergeben               & - Informationen sinvoll aufteilen\\
- Wichtigen Inhalt hervorheben & - Scrollen vermeiden\\
- Einfache Sprache                     &                                                    \\
\hline
\end{tabular}

\begin{tabular}{| l | l |}
\hline
 Metakognition & Navigation\\
\hline
\hline
- Zwischenziele                        & - Kerninhalte leicht zugänglich\\
- Ziele immer ersichtlich &  - Suchverlauf zeigen\\
- Checkliste                     & - Scrollen vemeiden\\
 - geringere Auswahlmöglichkeiten& - Links abkürzen \\
 - einheitliches und konsistentes Design&  - klare und eindeutige Kategorien \\
& - Schreibfehler ignorieren \\
& - Mischung aus Breiten- und Tiefensuche \\
\hline
\end{tabular}
\subsection{Einfache Sprache}
Schwierigkeit der Sprache ist messbar an Satzlänge, Wortlänge und Vokabular.\\
Bei einer Ordnung von 0 bis 12:\\
1 Geringer Wortschatz < 5 (Amerikanischer) Durchschnittsbürger < 12 Amtsprache\\
\begin{figure}
		\centering
		\includegraphics[width=\textwidth]{up_goer_five_part.png}
	\caption{Up Goer Five: Einfache Sprache aber schwierige Schrift}
	\end{figure}

\section{Beispiele}

\textbf{Invisque:}	http://www.invisque.mdx.ac.uk/
	\begin{itemize}
		\item Prototyp zur interaktiven Suche auf anderen Seiten
		\item Basierend auf Schreibtisch und Karteikarten Metapher
                        \item verzichtet nicht auf komplexer Sprache
                        \item kann inhalt des Gesuchten nicht beeinflussen
	\end{itemize}

\textbf{Lernportal ich-will-lernen.de} 
	\begin{itemize}
		\item Deutscher Volkshochschul-Verband e. V.
		\item Lernportal für verscheidene Zielgruppen
                        \item reduzierte Benutzeroberfläche
                        \item Sprachausgabe
		\item persönliche Unterstützung möglich
	\end{itemize}		 

\textbf{Text-freie Benutzereingabe für Analphabeten und semi-gebildete Benutzer} \\
 von Indrani Medhi, Aman Sagar und Kentaro Toyama 2006
	\begin{itemize}
		\item vollständig Textunabhängig
		\item Zielgruppe: Urban-Slums in Indien
                        \item Unterstützung für Job-Suche
                        \item Sprachausgabe
		\item Intuitiv
	\end{itemize}

\end{document}
