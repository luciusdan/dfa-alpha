
\frame{\frametitle{Erfahrungsbericht eines Analphabeten}
	
		Bernd Dahler (36) - [Zeitpunkt der Befragung unbekannt]\pause
		\begin{itemize}
			\item aufgewachsen mit 10 Geschwistern als 2. jüngstes Kind.\pause
			\item In der Schule:
			\begin{itemize}
				\item Beim Vorlesen gestottert und deshalb ausgelacht. \pause
				\item Sollte als Linkshänder mit der rechten Hand schreiben. \pause
				\item Auf der Hauptschule war er bei den Klassenarbeiten meistens krank. \pause
				\item Hat den Lehrstoff durch aufpassen im Unterricht mitbekommen. \pause
				\item Bekam einen Abschluss mit der Notiz, dass er nicht lesen und schreiben könne.			
			\end{itemize}
		\end{itemize}
		
	}
	
\frame{\frametitle{Erfahrungsbericht eines Analphabeten}
	
		\begin{itemize}
			\item Der Beruf:
			\begin{itemize}
				\item Bekam mit seinem Zeugnis eine Ausbildung als Galvaniseur. \pause 
					Ein Handwerksberuf, in dem man nicht viel lesen und schreiben muss.\pause
				\item Die Abschlussprüfung als mündliche Prüfung mit einer erhöhten Gebühr bestanden.			
			\end{itemize}
		\end{itemize}
		
	}
	
	\frame{\frametitle{Erfahrungsbericht eines Analphabeten}
	
		\begin{itemize}
			\item Das Privatleben:
			\begin{itemize}
				\item Im Restaurant wird immer Wienerschnitzel mit Pommes bestellt. \pause
				\item beim Lesen des Stadtplanes hatte er seine Brille vergessen \pause
				\item Eignet sich Allgemeinwissen durch Fernsehen an
			\end{itemize}
		\end{itemize}
		
	}