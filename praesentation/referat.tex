\documentclass{beamer}

\usepackage[utf8x]{inputenc}
\usepackage[T1]{fontenc}
\usepackage[ngerman]{babel}
\usepackage{amsmath,amsfonts,amssymb}
\usepackage{lmodern}
\usepackage{marvosym}

\usetheme{Berlin}

\begin{document}
    \author{Pascal Kn"uppel, Dirk Evers, Jan-Bernd Vosteen}
    \title{Design f"ur Analphabeten}
    \date{22.01.2013}
    
    \frame{\titlepage}
    
    \frame{\frametitle{Inhaltsverzeichnis}\tableofcontents}

	\subsection{Zielgruppe}

\frame{\frametitle{Zielgruppe}
	
		
	
	}
	

	\frame{\frametitle{Arten des Analphabetismus}
		\begin{itemize}
    			\item primärer Analphabetismus
			\item sekundärer Analphabetismus
			\item Semianalphabetismus
			\item funktionaler Analphabetismus
		\end{itemize}
	}


	
%Zahl der Betroffenen
	%weltweit
	%deutschland
	%vergleich zu anderen behinderungen

%Pers"onliche auswirkungen

%Was ist Analphabetismus?
	%analphabeten sind nicht dumm!
	%abstufungen von Analphabetismus
	%ursachen von analphabetismus

%Teilgruppen der Analphabeten
	%Legasteniker
	%lern schwache, lern behinderte
	%funktionale Analphabeten

%Demonstration wie sich analphabetismus anf"uhlt
	%Text oder Programm in fremder Sprache
		%Merkaartor in Niederl"andisch?

%Beeintr"achtigungen
	%lesen
	%errinerungsvermögen
	%"Denken "ubers Denken"
	%Orientierung (in dokumenten) und suche

%Design l"osungen
	%alphabetisierung.de
		%readspeaker.com
	%interactive visualization for low literacy users
		%nicht f"ur legastheniker
	%Textvermeidung
		% Zahlen werden von den meisten jedoch dennoch richtig identifiziert.

%Design testen
	%probanden finden
	%zusammenarbeit mit Probanden


\end{document}
