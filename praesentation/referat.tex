\documentclass{beamer}

\usepackage[utf8x]{inputenc}
\usepackage[T1]{fontenc}
\usepackage[ngerman]{babel}
\usepackage{amsmath,amsfonts,amssymb}
\usepackage{lmodern}
\usepackage{marvosym}

\usetheme{Berlin}

\begin{document}
    \author{Pascal Knüppel, Dirk Evers, Jan-Bernd Vosteen}
    \title{Design für Analphabeten}
    \date{22.01.2013}
    
    \frame{\titlepage}
    
    \frame{\frametitle{Inhaltsverzeichnis}\tableofcontents}

	\frame{\frametitle{Zielgruppe}
	
	}
	
%Zahl der Betroffenen
	%weltweit
	%deutschland
	%vergleich zu anderen behinderungen

%Persönliche auswirkungen

%Was ist Analphabetismus?
	%analphabeten sind nicht dumm!
	%abstufungen von Analphabetismus
	%ursachen von analphabetismus

%Teilgruppen der Analphabeten
	%Legasteniker
	%lern schwache, lern behinderte
	%funktionale Analphabeten

%Demonstration wie sich analphabetismus anfühlt
	%Text oder Programm in fremder Sprache
		%Merkaartor in Niederländisch?

%Beeinträchtigungen
	%lesen
	%errinerungsvermögen
	%"Denken übers Denken"
	%Orientierung (in dokumenten) und suche

%Design lösungen
	%alphabetisierung.de
		%readspeaker.com
	%interactive visualization for low literacy users
		%nicht für legastheniker

%Design testen
	%probanden finden
	%zusammenarbeit mit Probanden

\end{document}
